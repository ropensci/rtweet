\documentclass[a4paper]{book}
\usepackage[times,inconsolata,hyper]{Rd}
\usepackage{makeidx}
\usepackage[latin1]{inputenc} % @SET ENCODING@
% \usepackage{graphicx} % @USE GRAPHICX@
\makeindex{}
\begin{document}
\chapter*{}
\begin{center}
{\textbf{\huge Package `rtweet'}}
\par\bigskip{\large \today}
\end{center}
\begin{description}
\raggedright{}
\item[Type]\AsIs{Package}
\item[Title]\AsIs{Collecting Twitter Data}
\item[Version]\AsIs{0.4.8}
\item[Date]\AsIs{2017-07-19}
\item[Description]\AsIs{An implementation of calls designed to extract and organize Twitter data via
Twitter's REST and stream APIs. Functions formulate and send API requests, convert
response objects to more user friendly data structures---e.g., data frames---and
provide some aesthetically pleasing visualizations for exploring the data.}
\item[Depends]\AsIs{R (>= 3.1.0)}
\item[Imports]\AsIs{bit64,
httr (>= 1.0.0),
jsonlite,
magrittr,
methods,
openssl,
tibble}
\item[License]\AsIs{MIT + file LICENSE}
\item[LazyData]\AsIs{TRUE}
\item[URL]\AsIs{}\url{https://CRAN.R-project.org/package=rtweet}\AsIs{}
\item[BugReports]\AsIs{}\url{https://github.com/mkearney/rtweet/issues}\AsIs{}
\item[RoxygenNote]\AsIs{6.0.1}
\item[Suggests]\AsIs{ggplot2,
knitr,
rmarkdown,
testthat}
\item[VignetteBuilder]\AsIs{knitr}
\end{description}
\Rdcontents{\R{} topics documented:}
